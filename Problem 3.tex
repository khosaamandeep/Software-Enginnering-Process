\documentclass[a4paper, 11pt]{article}
\usepackage{comment} % enables the use of multi-line comments (\ifx \fi) 
\usepackage{fullpage} % changes the margin
\usepackage{hyperref}
\usepackage{mathtools}
\usepackage{amsmath}
\usepackage{amssymb}
\usepackage{booktabs} % For formal tables
\usepackage[ruled]{algorithm2e} % For algorithms
\renewcommand{\algorithmcfname}{ALGORITHM}
\pagenumbering{gobble}

\begin{document}\textbf{Problem 3: - }Collaboratively brainstorm and mind map with your team members to decide a
pseudocode format. The pseudocode format must be identical across the team. Select
algorithms for implementing your function and all its subordinate functions, if any.
Give technical reasons for selecting each of the algorithms, including their advantages
and disadvantages. (This, by reference, means that you must have at least two options to
choose from.)\par

\begin{enumerate}


\begin{algorithm}
begin: \\
1. verifyInputIsRealNumber(a,b,x)\\
2. Set temp = 1 \\ 
3. $\hspace{2em}$FOR COUNTER = 1 TO x \\
4. $\hspace{3em}$ UPDATE TEMP TO TEMP * b\\
5. counter = 0\\
6. temp2 = 1\\
7. $\hspace{2em}$ UPDATE temp2 TO a*temp2\\
8. PRINT TEMP2 \\
9. READ new value for X\\
10. REPEAT the algorithm for new value \\
end
\caption{ALGORITHM - NAME (a,b,x (input set))}
\end{algorithm}

\begin{algorithm}
begin:\\
1.IF  a,b,x $\in -\infty$  to  +$\infty$ \\
2. $\hspace{2em}$ continue with processing\\
3. else\\
4. $\hspace{2em}$ print error and take new inputs \\
end
\caption{verifyInputIsRealNumber(a,b,x)}
\end{algorithm}

\begin{enumerate}
    \item For Input : use READ
    \item For output : use PRINT
    \item For calculation : use COMPUTE
    \item For Initialize: use SET 
    \item For Add one: use INCREMENT 
\end{enumerate}
\\Advantages:-
\begin{enumerate}
  \item Exponential function is used to model population and help   coroners determine the time of death.
 \item  This function also assist in computing investments and used in radioactive decay.
 \end{enumerate}
 \\Disadvantages:-
\begin{enumerate}
\item Exponential functions are always either positive or negative.
\item The function lives entirely on one side or other side of the x-axis.
\end{enumerate}
\end{document}